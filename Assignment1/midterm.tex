\documentclass{article}
\usepackage{amsmath}
\usepackage{amssymb}
\begin{document}

\title{Parallel Processing Paradox: A Comparison of Amdhal's and Gustafson's Laws}
\author{Zak Kastl}
\date{\today}
\maketitle

\section{Introduction}
Processing power has advanced exponentially in the last 25 years. What once required room-sized supercomputers are now computable on the personal mobile phone. Computational processing has advanced so far, that creating smalller, faster processors beginning to become physically impossible. Processors generate heat during their workload. This heat reduces the efficency of the processor a little bit. With the increase of heate, eventually a single processor will not generate enough work to offset this heat. 

So how do we work around this? Current com

\subsection{Top Matter}
This subsection's content...

\subsubsection{Article Information}
This subsubsection's content...

\emph{hello} hello $x^2, \mathbb R, \mathcal C, \mathfrak F, \int_0^1 \sin(x) \mathrm d x$

\end{document}